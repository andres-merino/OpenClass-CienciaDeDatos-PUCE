%%%%%%%%%%%%%%%%%%%%%%%%%%%%%%%%%%%%%%%%%%%%%%%%%%%%%%%%
\fondo{celeste}
\section{Imágenes a color}
\fondo{blanco}
%%%%%%%%%%%%%%%%%%%%%%%%%%%%%%%%%%%%%%%%%%%%%%%%%%%%%%%%


%%%%%%%%%%%%%%%%%%%%%%%%%%%%%%%%%%%%%%%%%%%%%%%%%%%%%%%%
%%%%%%%%%%%%%%%%%%%%%%%%%%%%%%%%%%%%%%%%%%%%%%%%%%%%%%%%
\begin{frame}{El formato RGB}

    \begin{columns}
    \column{8cm}
        Es un modelo de color basado en la síntesis aditiva, con el que es posible representar un color mediante la mezcla por adición de los tres colores de luz primarios: \textcolor{red}{rojo}, \textcolor{green}{verde} o \textcolor{blue}{azul}
    \column{5cm}
        \postitimg[5cm]{Figuras/Fig08}
    \end{columns}

\end{frame}


%%%%%%%%%%%%%%%%%%%%%%%%%%%%%%%%%%%%%%%%%%%%%%%%%%%%%%%%
%%%%%%%%%%%%%%%%%%%%%%%%%%%%%%%%%%%%%%%%%%%%%%%%%%%%%%%%
\begin{frame}{Ejemplo}
    \[
        R = \begin{pmatrix}
            0.0& 0.3& 0.9\\
            1.0& 0.0& 1.0
        \end{pmatrix};\quad
        G = \begin{pmatrix}
            0.0& 0.3& 0.3\\
            0.0& 1.0& 1.0
        \end{pmatrix};\quad
        B = \begin{pmatrix}
            0.0& 0.3& 0.9\\
            0.0& 0.0& 1.0
        \end{pmatrix}.
    \]
    \pause  
    \begin{center}
    \postitimg[4cm]{Figuras/Fig09}
    \end{center}

\end{frame}

%%%%%%%%%%%%%%%%%%%%%%%%%%%%%%%%%%%%%%%%%%%%%%%%%%%%%%%%
%%%%%%%%%%%%%%%%%%%%%%%%%%%%%%%%%%%%%%%%%%%%%%%%%%%%%%%%
\begin{frame}{Jueguemos con Mario}
    
    \begin{center}
    \postitimg[3.5cm]{Figuras/Mario}
    \end{center}

\end{frame}
