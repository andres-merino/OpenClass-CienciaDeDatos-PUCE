%%%%%%%%%%%%%%%%%%%%%%%%%%%%%%%%%%%%%%%%%%%%%%%%%%%%%%%%
\fondo{celeste}
\section{Matrices}
\fondo{blanco}
%%%%%%%%%%%%%%%%%%%%%%%%%%%%%%%%%%%%%%%%%%%%%%%%%%%%%%%%


%%%%%%%%%%%%%%%%%%%%%%%%%%%%%%%%%%%%%%%%%%%%%%%%%%%%%%%%
%%%%%%%%%%%%%%%%%%%%%%%%%%%%%%%%%%%%%%%%%%%%%%%%%%%%%%%%
\begin{frame}

    \begin{block}{Matrices}
        Una matriz es un arreglo de números de la forma:
        \[
            A = 
            \begin{pmatrix}
            a_{0,0} & a_{0,1} & \cdots & a_{0,n-1} \\ 
            a_{1,0} & a_{1,1} & \cdots & a_{1,n-1} \\ 
            \vdots & \vdots & \ddots & \vdots\\ 
            a_{m-1,0} & a_{m-1,1} & \cdots & a_{m-1,n-1}
            \end{pmatrix}.
        \]
        Con esto, se dice que $A$ es de orden $m \times n$ y que tiene $m$ filas y $n$ columnas. Al conjunto de todas las matrices de orden $m \times n$ se denota por $\Mat[\R]{m}{n}$.
    \end{block}

\end{frame}


%%%%%%%%%%%%%%%%%%%%%%%%%%%%%%%%%%%%%%%%%%%%%%%%%%%%%%%%
%%%%%%%%%%%%%%%%%%%%%%%%%%%%%%%%%%%%%%%%%%%%%%%%%%%%%%%%
\begin{frame}{Ejemplos}

\end{frame}


%%%%%%%%%%%%%%%%%%%%%%%%%%%%%%%%%%%%%%%%%%%%%%%%%%%%%%%%
%%%%%%%%%%%%%%%%%%%%%%%%%%%%%%%%%%%%%%%%%%%%%%%%%%%%%%%%
% \begin{frame}[fragile]

%     \titulo{¿Puedo hacer esto en la compu?}

% \begin{pycodigo}
% \begin{ipynbcodigo}
% \begin{lstlisting}[language=Python]
% from numpy import np
% \end{lstlisting}
% \end{ipynbcodigo}
% \begin{ipynbcodigo}
% \begin{lstlisting}[language=Python]
% A = np.array([
%     [1, 0, 1],
%     [0, 1, 0]
%     ])
% print(A)
% \end{lstlisting}
% \end{ipynbcodigo}
% \end{pycodigo}

% \end{frame}


%%%%%%%%%%%%%%%%%%%%%%%%%%%%%%%%%%%%%%%%%%%%%%%%%%%%%%%%
%%%%%%%%%%%%%%%%%%%%%%%%%%%%%%%%%%%%%%%%%%%%%%%%%%%%%%%%
\begin{frame}

    \begin{block}{Suma de matrices}
        Dadas dos matrices $A$ y $B$ del mismo orden, se define la suma de matrices, $A+B$, a la matriz que resulta de sumar, componente a componete, las componentes de las matrices.
    \end{block}
    \vspace{3cm}

\end{frame}

% %%%%%%%%%%%%%%%%%%%%%%%%%%%%%%%%%%%%%%%%%%%%%%%%%%%%%%%%
% %%%%%%%%%%%%%%%%%%%%%%%%%%%%%%%%%%%%%%%%%%%%%%%%%%%%%%%%
% \begin{frame}

%     \begin{block}{Multiplicación de matrices por un escalar}
%         Dada una matriz $A$ y un número real $\alpha$, se define la multiplicación de una matriz por un escalar, $\alpha A$, a la matriz que resulta de multiplicar, cada componete, por el número $\alpha$.
%     \end{block}
%     \vspace{3cm}

% \end{frame}